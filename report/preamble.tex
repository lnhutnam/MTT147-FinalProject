% basics
\usepackage[utf8]{inputenc}
\usepackage[vietnamese]{babel}

\usepackage[top=3.5cm, bottom=3.0cm, left=3.5cm, right=2.0cm]{geometry} % căn lề theo quy chuẩn KLTN
\usepackage{indentfirst}
\parskip 2pt
\headsep=12pt
\linespread{1.45}
\usepackage{titlesec}
\titleformat{\chapter}[display]{\bfseries \large \center}{CHƯƠNG \thechapter}{0.3em}{}[]
\titleformat{\section}{\bfseries}{ \thesection.}{0.3em}{}[]
\titleformat{\subsection}{\it \bfseries }{ \thesubsection.}{0.3em}{}[]
\titleformat{\subsubsection}{ \it }{ \thesubsubsection.}{0.3em}{}[]
\titlespacing{\chapter}{1em}{0.1em}{3em}

\usepackage{faktor}
\usepackage{authblk} % Add this package

\usepackage{textcomp}
% \usepackage[dutch]{babel}
% \usepackage{url}
% \usepackage{hyperref}
% \hypersetup{
%     colorlinks,
%     linkcolor={black},
%     citecolor={black},
%     urlcolor={blue!80!black}
% }
\usepackage{graphicx}
\usepackage{float}
\usepackage{booktabs}
\usepackage{enumitem}
% \usepackage{parskip}
\usepackage{emptypage}
\usepackage{subcaption}
\usepackage{multicol}
\usepackage[usenames,dvipsnames]{xcolor}

% \usepackage{cmbright}


\usepackage{amsmath, amsfonts, mathtools, amsthm, amssymb}
\usepackage{mathrsfs}
\usepackage{cancel}
\usepackage{bm}
\newcommand\N{\ensuremath{\mathbb{N}}}
\newcommand\R{\ensuremath{\mathbb{R}}}
\newcommand\Z{\ensuremath{\mathbb{Z}}}
\renewcommand\O{\ensuremath{\emptyset}}
\newcommand\Q{\ensuremath{\mathbb{Q}}}
\newcommand\C{\ensuremath{\mathbb{C}}}
\DeclareMathOperator{\sgn}{sgn}
\usepackage{systeme}
\let\svlim\lim\def\lim{\svlim\limits}
\let\implies\Rightarrow
\let\impliedby\Leftarrow
\let\iff\Leftrightarrow
\let\epsilon\varepsilon
\usepackage{stmaryrd} % for \lightning
\newcommand\contra{\scalebox{1.1}{$\lightning$}}
% \let\phi\varphi





% correct
\definecolor{correct}{HTML}{009900}
\newcommand\correct[2]{\ensuremath{\:}{\color{red}{#1}}\ensuremath{\to }{\color{correct}{#2}}\ensuremath{\:}}
\newcommand\green[1]{{\color{correct}{#1}}}



% horizontal rule
\newcommand\hr{
    \noindent\rule[0.5ex]{\linewidth}{0.5pt}
}


% hide parts
\newcommand\hide[1]{}



% si unitx
\usepackage{siunitx}
\sisetup{locale = FR}
% \renewcommand\vec[1]{\mathbf{#1}}
\newcommand\mat[1]{\mathbf{#1}}


% tikz
\usepackage{tikz}
\usepackage{tikz-cd}
\usetikzlibrary{intersections, angles, quotes, calc, positioning}
\usetikzlibrary{arrows.meta}
\usepackage{pgfplots}
\pgfplotsset{compat=1.13}


\tikzset{
    force/.style={thick, {Circle[length=2pt]}-stealth, shorten <=-1pt}
}

% theorems
\makeatother
\usepackage{thmtools}
\usepackage[framemethod=TikZ]{mdframed}
\mdfsetup{skipabove=1em,skipbelow=0em}


\theoremstyle{definition}

\declaretheoremstyle[
    headfont=\bfseries\sffamily\color{ForestGreen!70!black}, bodyfont=\normalfont,
    mdframed={
        linewidth=2pt,
        rightline=false, topline=false, bottomline=false,
        linecolor=ForestGreen, backgroundcolor=ForestGreen!5,
    }
]{thmgreenbox}

\declaretheoremstyle[
    headfont=\bfseries\sffamily\color{NavyBlue!70!black}, bodyfont=\normalfont,
    mdframed={
        linewidth=2pt,
        rightline=false, topline=false, bottomline=false,
        linecolor=NavyBlue, backgroundcolor=NavyBlue!5,
    }
]{thmbluebox}

\declaretheoremstyle[
    headfont=\bfseries\sffamily\color{NavyBlue!70!black}, bodyfont=\normalfont,
    mdframed={
        linewidth=2pt,
        rightline=false, topline=false, bottomline=false,
        linecolor=NavyBlue
    }
]{thmblueline}

\declaretheoremstyle[
    headfont=\bfseries\sffamily\color{RawSienna!70!black}, bodyfont=\normalfont,
    mdframed={
        linewidth=2pt,
        rightline=false, topline=false, bottomline=false,
        linecolor=RawSienna, backgroundcolor=RawSienna!5,
    }
]{thmredbox}

\declaretheoremstyle[
    headfont=\bfseries\sffamily\color{RawSienna!70!black}, bodyfont=\normalfont,
    numbered=no,
    mdframed={
        linewidth=2pt,
        rightline=false, topline=false, bottomline=false,
        linecolor=RawSienna, backgroundcolor=RawSienna!1,
    },
    qed=\qedsymbol
]{thmproofbox}

\declaretheoremstyle[
    headfont=\bfseries\sffamily\color{NavyBlue!70!black}, bodyfont=\normalfont,
    numbered=no,
    mdframed={
        linewidth=2pt,
        rightline=false, topline=false, bottomline=false,
        linecolor=NavyBlue, backgroundcolor=NavyBlue!1,
    },
]{thmexplanationbox}



% \declaretheoremstyle[headfont=\bfseries\sffamily, bodyfont=\normalfont, mdframed={ nobreak } ]{thmgreenbox}
% \declaretheoremstyle[headfont=\bfseries\sffamily, bodyfont=\normalfont, mdframed={ nobreak } ]{thmredbox}
% \declaretheoremstyle[headfont=\bfseries\sffamily, bodyfont=\normalfont]{thmbluebox}
% \declaretheoremstyle[headfont=\bfseries\sffamily, bodyfont=\normalfont]{thmblueline}
% \declaretheoremstyle[headfont=\bfseries\sffamily, bodyfont=\normalfont, numbered=no, mdframed={ rightline=false, topline=false, bottomline=false, }, qed=\qedsymbol ]{thmproofbox}
% \declaretheoremstyle[headfont=\bfseries\sffamily, bodyfont=\normalfont, numbered=no, mdframed={ nobreak, rightline=false, topline=false, bottomline=false } ]{thmexplanationbox}

\declaretheorem[style=thmgreenbox, name=Definition]{definition}
\declaretheorem[style=thmbluebox, numbered=no, name=Example]{eg}
\declaretheorem[style=thmredbox, name=Proposition]{prop}
\declaretheorem[style=thmredbox, name=Theorem]{theorem}
\declaretheorem[style=thmredbox, name=Lemma]{lemma}
\declaretheorem[style=thmredbox, numbered=no, name=Corollary]{corollary}

\declaretheorem[style=thmproofbox, name=Proof]{replacementproof}
\renewenvironment{proof}[1][\proofname]{\vspace{-10pt}\begin{replacementproof}}{\end{replacementproof}}


\declaretheorem[style=thmexplanationbox, name=Proof]{tmpexplanation}
\newenvironment{explanation}[1][]{\vspace{-10pt}\begin{tmpexplanation}}{\end{tmpexplanation}}

\declaretheorem[style=thmblueline, numbered=no, name=Remark]{remark}
\declaretheorem[style=thmblueline, numbered=no, name=Note]{note}

\newtheorem*{uovt}{UOVT}
\newtheorem*{notation}{Notation}
\newtheorem*{previouslyseen}{As previously seen}
\newtheorem*{problem}{Problem}
\newtheorem*{observe}{Observe}
\newtheorem*{property}{Property}
\newtheorem*{intuition}{Intuition}


\usepackage{etoolbox}
\AtEndEnvironment{vb}{\null\hfill$\diamond$}%
\AtEndEnvironment{intermezzo}{\null\hfill$\diamond$}%
% \AtEndEnvironment{opmerking}{\null\hfill$\diamond$}%

% http://tex.stackexchange.com/questions/22119/how-can-i-change-the-spacing-before-theorems-with-amsthm
\makeatletter
% \def\thm@space@setup{%
%   \thm@preskip=\parskip \thm@postskip=0pt
% }

\newcommand{\oefening}[1]{%
    \def\@oefening{#1}%
    \subsection*{Oefening #1}
}

\newcommand{\suboefening}[1]{%
    \subsubsection*{Oefening \@oefening.#1}
}

\newcommand{\exercise}[1]{%
    \def\@exercise{#1}%
    \subsection*{Exercise #1}
}

\newcommand{\subexercise}[1]{%
    \subsubsection*{Exercise \@exercise.#1}
}


\usepackage{xifthen}

\def\testdateparts#1{\dateparts#1\relax}
\def\dateparts#1 #2 #3 #4 #5\relax{
    \marginpar{\small\textsf{\mbox{#1 #2 #3 #5}}}
}

\def\@lesson{}%
\newcommand{\lesson}[3]{
    \ifthenelse{\isempty{#3}}{%
        \def\@lesson{Lecture #1}%
    }{%
        \def\@lesson{Lecture #1: #3}%
    }%
    \subsection*{\@lesson}
    \testdateparts{#2}
}

% \renewcommand\date[1]{\marginpar{#1}}


% fancy headers
\usepackage{fancyhdr}
% \pagestyle{fancy}

% \fancyhead[LE,RO]{Gilles Castel}
% \fancyhead[RO,LE]{\@lesson}
% \fancyhead[RE,LO]{}
% \fancyfoot[LE,RO]{\thepage}
% \fancyfoot[C]{\leftmark}

\makeatother




% notes
\usepackage{todonotes}
\usepackage{tcolorbox}

\tcbuselibrary{breakable}
\newenvironment{verbetering}{\begin{tcolorbox}[
    arc=0mm,
    colback=white,
    colframe=green!60!black,
    title=Opmerking,
    fonttitle=\sffamily,
    breakable
]}{\end{tcolorbox}}

\newenvironment{noot}[1]{\begin{tcolorbox}[
    arc=0mm,
    colback=white,
    colframe=white!60!black,
    title=#1,
    fonttitle=\sffamily,
    breakable
]}{\end{tcolorbox}}




% figure support
\usepackage{import}
\usepackage{xifthen}
\pdfminorversion=7
\usepackage{pdfpages}
\usepackage{transparent}
\newcommand{\incfig}[1]{%
    \def\svgwidth{\columnwidth}
    \import{./figures/}{#1.pdf_tex}
}

% %http://tex.stackexchange.com/questions/76273/multiple-pdfs-with-page-group-included-in-a-single-page-warning
\pdfsuppresswarningpagegroup=1


\newcommand\contents{
	\renewcommand*\contentsname{MỤC LỤC}
	\newpage
	\phantomsection
	\linespread{1.25}
	% \addcontentsline{toc}{chapter}{{\bf MỤC LỤC\rm }}
	\tableofcontents%
	\linespread{1.45}
}
\newcommand\listImages{
	\renewcommand*{\listfigurename}{\bfseries DANH MỤC CÁC HÌNH VẼ, ĐỒ THỊ}
	\newpage
	\phantomsection
	\addcontentsline{toc}{chapter}{{\bf DANH MỤC CÁC HÌNH VẼ, ĐỒ THỊ\rm }}

	{% Thêm chữ hình vẽ
		\let\oldnumberline\numberline%
		\renewcommand{\numberline}{Hình~\oldnumberline}%
		\listoffigures%
	}
}

\newcommand\listTables{
	\renewcommand*{\listtablename}{\bfseries DANH MỤC CÁC BẢNG}

	{
		\let\oldnumberline\numberline%
		\renewcommand{\numberline}{Bảng~\oldnumberline}%
		\newpage
		\phantomsection
		\addcontentsline{toc}{chapter}{{\bf DANH MỤC CÁC BẢNG \rm}}
		\listoftables%

	}
}

\usepackage{listings}

\lstset{literate=%
% Vần a
	{á}{{\'a}}1
	{à}{{\`a}}1
	{ạ}{{\d a}}1
	{ả}{{\h a}}1
	{ã}{{\~ a}}1
	%
	{Á}{{\'A}}1
	{À}{{\`A}}1
	{Ạ}{{\d A}}1
	{Ả}{{\h A}}1
	{Ã}{{\~ A}}1
%
% Vần ă
	{ă}{{\u a}}1
	{ắ}{{\'\abreve }}1
	{ằ}{{\`\abreve }}1
	{ặ}{{\d \abreve }}1
	{ẳ}{{\h \abreve }}1
	{ẵ}{{\~\abreve }}1
	%
	{Ă}{{\u A}}1
	{Ắ}{{\'\ABREVE }}1
	{Ằ}{{\`\ABREVE }}1
	{Ặ}{{\d \ABREVE }}1
	{Ẳ}{{\h \ABREVE }}1
	{Ẵ}{{\~\ABREVE }}1
%
% Vần â
	{â}{{\^ a}}1
	{ấ}{{\'\acircumflex }}1
	{ầ}{{\`\acircumflex }}1
	{ậ}{{\d \acircumflex }}1
	{ẩ}{{\h \acircumflex }}1
	{ẫ}{{\~\acircumflex }}1
	 %
	{Â}{{\^ A}}1
	{Ấ}{{\'\ACIRCUMFLEX }}1
	{Ầ}{{\`\ACIRCUMFLEX }}1
	{Ậ}{{\d \ACIRCUMFLEX }}1
	{Ẩ}{{\h \ACIRCUMFLEX }}1
	{Ẫ}{{\~\ACIRCUMFLEX }}1
%
% Vần đ
	{đ}{{\dj }}1
	{Đ}{{\DJ }}1
%
% Vần e
	{é}{{\'e}}1
	{è}{{\`e}}1
	{ẹ}{{\d e}}1
	{ẻ}{{\h e}}1
	{ẽ}{{\~ e}}1
	%
	{É}{{\'E}}1
	{È}{{\`E}}1
	{Ẹ}{{\d E}}1
	{Ẻ}{{\h E}}1
	{Ẽ}{{\~ E}}1
%
% Vần ê
	{ê}{{\^e}}1
	{ế}{{\'\ecircumflex }}1
	{ề}{{\`\ecircumflex }}1
	{ệ}{{\d \ecircumflex }}1
	{ể}{{\h \ecircumflex }}1
	{ễ}{{\~\ecircumflex }}1
	%
	{Ê}{{\^E}}1
	{Ế}{{\'\ECIRCUMFLEX }}1
	{Ề}{{\`\ECIRCUMFLEX }}1
	{Ệ}{{\d \ECIRCUMFLEX }}1
	{Ể}{{\h \ECIRCUMFLEX }}1
	{Ễ}{{\~\ECIRCUMFLEX }}1
%
% Vần i
	{í}{{\'i}}1
	{ì}{{\`\i }}1
	{ị}{{\d i}}1
	{ỉ}{{\h i}}1
	{ĩ}{{\~\i }}1
	%
	{Í}{{\'I}}1
	{Ì}{{\`I}}1
	{Ị}{{\d I}}1
	{Ỉ}{{\h I}}1
	{Ĩ}{{\~I}}1
%
% Vần o
	{ó}{{\'o}}1
	{ò}{{\`o}}1
	{ọ}{{\d o}}1
	{ỏ}{{\h o}}1
	{õ}{{\~o}}1
	%
	{Ó}{{\'O}}1
	{Ò}{{\`O}}1
	{Ọ}{{\d O}}1
	{Ỏ}{{\h O}}1
	{Õ}{{\~O}}1
%
% Vần ô
	{ô}{{\^o}}1
	{ố}{{\'\ocircumflex }}1
	{ồ}{{\`\ocircumflex }}1
	{ộ}{{\d \ocircumflex }}1
	{ổ}{{\h \ocircumflex }}1
	{ỗ}{{\~\ocircumflex }}1
	%
	{Ô}{{\^O}}1
	{Ố}{{\'\OCIRCUMFLEX }}1
	{Ồ}{{\`\OCIRCUMFLEX }}1
	{Ộ}{{\d \OCIRCUMFLEX }}1
	{Ổ}{{\h \OCIRCUMFLEX }}1
	{Ỗ}{{\~\OCIRCUMFLEX }}1
%
% Vần ơ
	{ơ}{{\ohorn }}1
	{ớ}{{\'\ohorn }}1
	{ờ}{{\`\ohorn }}1
	{ợ}{{\d \ohorn }}1
	{ở}{{\h \ohorn }}1
	{ỡ}{{\~\ohorn }}1
	%
	{Ơ}{{\OHORN }}1
	{Ớ}{{\'\OHORN }}1
	{Ờ}{{\`\OHORN }}1
	{Ợ}{{\d \OHORN }}1
	{Ở}{{\h \OHORN }}1
	{Ỡ}{{\~\OHORN }}1
%
% Vần u
	{ú}{{\'u}}1
	{ù}{{\`u}}1
	{ụ}{{\d u}}1
	{ủ}{{\h u}}1
	{ũ}{{\~u}}1
	%
	{Ú}{{\'U}}1
	{Ù}{{\`U}}1
	{Ụ}{{\d U}}1
	{Ủ}{{\h U}}1
	{Ũ}{{\~U}}1
%
% Vần ư
	{ư}{{\uhorn }}1
	{ứ}{{\'\uhorn }}1
	{ừ}{{\`\uhorn }}1
	{ự}{{\d \uhorn }}1
	{ử}{{\h \uhorn }}1
	{ữ}{{\~\uhorn }}1
	%
	{Ư}{{\UHORN }}1
	{Ứ}{{\'\UHORN }}1
	{Ừ}{{\`\UHORN }}1
	{Ự}{{\d \UHORN }}1
	{Ử}{{\h \UHORN }}1
	{Ữ}{{\~\UHORN }}1
%
% Vần y
	{ý}{{\'y}}1
	{ỳ}{{\`y}}1
	{ỵ}{{\d y}}1
	{ỷ}{{\h y}}1
	{ỹ}{{\~y}}1
	%
	{Ý}{{\'Y}}1
	{Ỳ}{{\`Y}}1
	{Ỵ}{{\d Y}}1
	{Ỷ}{{\h Y}}1
	{Ỹ}{{\~Y}}1
}

\usepackage{color}
\definecolor{codebg}{rgb}{0.95,0.95,0.95}
\definecolor{mygreen}{rgb}{0,0.6,0}

% Define listing settings for R
\lstset{
    language=R,
    backgroundcolor=\color{codebg},
    basicstyle=\ttfamily,
    keywordstyle=\color{blue},
    commentstyle=\color{mygreen},
    stringstyle=\color{red},
    showstringspaces=false,
    breaklines=true,
    frame=single,
    rulecolor=\color{black},
    xleftmargin=2em,
    xrightmargin=2em,
    numbers=left,           % Add line numbers
    numberstyle=\tiny,      % Style of the line numbers
    stepnumber=1,           % Number every line
    numbersep=5pt           % Space between line numbers and code
}

\usepackage{minted}
\usepackage{longtable}

\newcommand\refs{
	\chapter*{TÀI LIỆU THAM KHẢO}%
	\addcontentsline{toc}{chapter}{{\bf TÀI LIỆU THAM KHẢO}}
	\phantomsection
	\titleformat{\chapter}{\bfseries  \large}{CHƯƠNG \thechapter.}{}{}[]
	\titlespacing{\chapter}{0.1em}{0.1em}{1.1em}
	\printbibliography[keyword={Vietnam},heading=subbibliography,title={Tiếng Việt:}]
	\printbibliography[notkeyword={Vietnam},heading=subbibliography,title={Tiếng Anh:}]
}