\chapter{TỔNG KẾT ĐỒ ÁN}


\section{Kết luận}

Trong quá trình thực hiện phân tích ANOVA, chúng tôi đã tiến hành các bước cần thiết để đảm bảo tính chính xác và tin cậy của kết quả. Cụ thể:

\begin{itemize}
    \item \textbf{Kiểm tra các giả định cơ bản}: Các giả định về độc lập, phân phối chuẩn của dữ liệu, và đồng nhất phương sai đã được kiểm tra kỹ lưỡng. Điều này đảm bảo rằng dữ liệu đáp ứng các điều kiện cần thiết để thực hiện ANOVA.
    
    \item \textbf{Thiết kế nghiên cứu hợp lý}: Chúng tôi đã xác định và thực hiện phân tích với số lượng nhóm và mẫu phù hợp. Sự ngẫu nhiên trong phân bổ mẫu vào các nhóm giúp giảm thiểu sai lệch và nâng cao tính chính xác của kết quả.
    
    \item \textbf{Phân tích dữ liệu}: ANOVA được thực hiện để xác định sự khác biệt có ý nghĩa giữa các nhóm. Khi phát hiện sự khác biệt, chúng tôi đã tiến hành các kiểm định hậu hoc để làm rõ cặp nhóm nào có sự khác biệt cụ thể. Việc kiểm tra tương tác giữa các yếu tố cũng được thực hiện để hiểu rõ hơn về mối quan hệ giữa các biến.
    
    \item \textbf{Xử lý ngoại lệ}: Chúng tôi đã xác định và xử lý các điểm ngoại lai, thực hiện phân tích ANOVA với và không có các ngoại lệ này để đảm bảo tính toàn vẹn của kết quả.
    
    \item \textbf{Kết quả và diễn giải}: Kết quả của phân tích ANOVA đã chỉ ra sự khác biệt có ý nghĩa giữa các nhóm nghiên cứu. Chúng tôi đã diễn giải kết quả này một cách rõ ràng, kết hợp với các số liệu về độ lớn hiệu ứng để cung cấp cái nhìn toàn diện về ảnh hưởng của các yếu tố lên biến phụ thuộc.
    
    \item \textbf{Hạn chế và đề xuất}: Dù kết quả phân tích ANOVA là hợp lý và có giá trị, chúng tôi cũng đã thảo luận về các hạn chế của nghiên cứu, chẳng hạn như kích thước mẫu và các giả định không hoàn toàn được đáp ứng. Các đề xuất cho nghiên cứu tiếp theo đã được đưa ra nhằm cải thiện tính chính xác và khả năng áp dụng của mô hình.
\end{itemize}